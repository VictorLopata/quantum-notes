% Theorem
\newtcbtheorem[number within = section]{theorem}{Theorem}%
{enhanced,frame empty,interior empty,colframe=TheoremColor!50!white,
	coltitle=TheoremColor!50!black,fonttitle=\bfseries,colbacktitle=TheoremColor!15!white,
	borderline={0.5mm}{0mm}{TheoremColor!15!white},
	borderline={0.5mm}{0mm}{TheoremColor!50!white},
	attach boxed title to top left={yshift=-2mm, xshift=3mm},
	boxed title style={boxrule=0.4pt},varwidth boxed title}{theo}

% Definition
\newtcbtheorem[number within = section]{definition}{Definition}%
{enhanced,frame empty,interior empty,colframe=DefColor!50!white,
	coltitle=DefColor!50!black,fonttitle=\bfseries,colbacktitle=DefColor!15!white,
	borderline={0.5mm}{0mm}{DefColor!15!white},
	borderline={0.5mm}{0mm}{DefColor!50!white},
	attach boxed title to top left={yshift=-2mm, xshift=3mm},
	boxed title style={boxrule=0.4pt},varwidth boxed title}{defo}

% Corollary
\newtcbtheorem[number within = section]{corollary}{Corollary}%
{enhanced,frame empty,interior empty,colframe=CorollaryColor!50!white,
	coltitle=CorollaryColor!50!black,fonttitle=\bfseries,colbacktitle=CorollaryColor!15!white,
	borderline={0.5mm}{0mm}{CorollaryColor!15!white},
	borderline={0.5mm}{0mm}{CorollaryColor!50!white},
	attach boxed title to top left={yshift=-2mm, xshift=3mm},
	boxed title style={boxrule=0.4pt},varwidth boxed title}{defo}
 
% Osservazione
\newtcbtheorem[number within = section]{oss}{Observation}%
{enhanced,frame empty,interior empty,colframe=CorollaryColor!50!white,
	coltitle=CorollaryColor!50!black,fonttitle=\bfseries,colbacktitle=CorollaryColor!15!white,
	borderline={0.5mm}{0mm}{CorollaryColor!15!white},
	borderline={0.5mm}{0mm}{CorollaryColor!50!white},
	attach boxed title to top left={yshift=-2mm, xshift=3mm},
	boxed title style={boxrule=0.4pt},varwidth boxed title}{defo}
 
% Example
\newtcbtheorem[number within = section]{example}{Example}%
{enhanced,frame empty,interior empty,colframe=ExampleColor!50!white,
	coltitle=ExampleColor!50!black,fonttitle=\bfseries,colbacktitle=ExampleColor!15!white,
	borderline={0.5mm}{0mm}{ExampleColor!15!white},
	borderline={0.5mm}{0mm}{ExampleColor!50!white},
	attach boxed title to top left={yshift=-2mm, xshift=3mm},
	boxed title style={boxrule=0.4pt},varwidth boxed title}{defo}

% Proof
\tcolorboxenvironment{proof}{% `proof' from `amsthm'
	blanker,breakable,left=5mm,
	before skip=10pt,after skip=10pt,
	borderline west={1mm}{0pt}{ProofColor!50!white}}

% Fact
\newtcbtheorem[number within = section]{fact}{Fact}%
{enhanced,frame empty,interior empty,colframe=ExampleColor!50!white,
	coltitle=ExampleColor!50!black,fonttitle=\bfseries,colbacktitle=ExampleColor!15!white,
	borderline={0.5mm}{0mm}{FactColor!15!white},
	borderline={0.5mm}{0mm}{FactColor!50!white},
	attach boxed title to top left={yshift=-2mm, xshift=3mm},
	boxed title style={boxrule=0.4pt},varwidth boxed title}{facto}
\section{Nozioni Matematiche}
\subsection{Strutture algebriche}
\begin{definition}{Struttura Algebrica}{}
    Definiamo come \textbf{struttura algebrica} un insieme munito di una o più operazioni. Spesso viene indicato con la notazione $(A, m)$, dove $A$ è l'insieme ed $m$ è l'operazione.
\end{definition}

\begin{definition}{Principali strutture algebriche}{}
    Sia $(A,m)$ una struttura algebrica, dove $A$ è l'insieme ed $m$ è un'operazione binaria chiusa sull'insieme. Tale struttura può essere definita come:
    \begin{itemize}
        \item \textbf{Semigruppo}: se $m$ è \underline{associativa}.
        \item \textbf{Monoide}: se $m$ è \underline{associativa} e munita dell'\underline{elemento neutro}.
        \item \textbf{Gruppo}: se $m$ è \underline{associativa}, munita dell'\underline{elemento neutro} e dell'\underline{elemento inverso}.
        \item \textbf{Gruppo abeliano}: se $m$ è \underline{associativa}, munita dell'\underline{elemento neutro} e dell'\underline{inverso} ed è \underline{commutativa}.
    \end{itemize}
\end{definition}

\begin{definition}{Anello}{}
    Sia $(A,+,\cdot)$ una struttura algebrica. Possiamo definirla come \textbf{anello} se:
    \begin{itemize}
        \item $(A,+)$ è un \textbf{gruppo abeliano}.
        \item $(A,\cdot)$ è un \textbf{semigruppo}.
        \item La moltiplicazione è distributiva rispetto alla somma:
        \begin{equation}
            \begin{split}
                a \cdot (b + c) = (a \cdot b) + (a \cdot c) \\
                (a + b) \cdot c= (a \cdot c) + (b \cdot c)   
            \end{split}
        \end{equation}
    \end{itemize}
    Possiamo definirlo anche come \textbf{anello commutativo} se $(A,\cdot)$ è munita della commutatività.
\end{definition}
\begin{fact}{}{}
    Sia $(A,+,\cdot)$ un anello. Allora:
    \begin{equation}
        \forall x,y\in A \quad (xy)^{-1} = y^{-1}x^{-1}
    \end{equation}
\end{fact}
\begin{definition}{Campo}{}
    Sia $(K,+,\cdot)$ una struttura algebrica. Possiamo deifinirla come \textbf{campo} se:
    \begin{itemize}
        \item $(K,+,\cdot)$ è un \textbf{anello commutativo}.
        \item $(K \backslash 0,\cdot)$ è un \textbf{gruppo abeliano}.
    \end{itemize}
\end{definition}

\subsection{Numeri complessi}
\subsection{Spazi Vettoriali}
\begin{definition}{Norma Euclidiana}{}
    Sia $v$ un vettore avente numeri complessi come entrate:
    \begin{equation}
        v = \left(
        \begin{array}{c}
             \alpha_1  \\
             \vdots \\
            \alpha_n   
        \end{array}\right)
    \end{equation}
    Definiamo la sua \textbf{norma Euclidiana} come:
    \begin{equation}
        \Vert v \Vert = \sqrt{\sum_{k=1}^{n} |{\alpha_k}^2|}
    \end{equation}
\end{definition}
\subsection{Matrici}
\begin{definition}{Trasposta di una matrice}{}
    Sia $A$ una matrice. Definiamo come \textbf{matrice trasposta} di $A$, rappresentata dal simbolo $A^{\mathrm{T}}$, come la matrice avente il cui generico elemento con indici $(i,j)$ è l'elemento con indice $(j,i)$ della matrice originaria.
    In altre parole, la matrice trasposta di una matrice è la matrice ottenuta scambiandone le righe con le colonne.
\end{definition}
\begin{example}{}{}
\begin{itemize}
    \item $A = \left(\begin{array}{ccc}
        2 & 1 & 4 \\
        0 & 0 & 3
    \end{array}\right) \quad A^{\mathrm{T}} = \left(\begin{array}       {cc}
        2 & 0 \\
        1 & 0 \\
        4 & 3
    \end{array}\right)$
    \item $A = \left(\begin{array}{ccccc}
        1 & 2 & 3 & 4 & 5 \\
        6 & 7 & 8 & 9 & 10 \\
        11 & 12 & 13 & 14 & 15 \\
        16 & 17 & 18 & 19 & 20 
    \end{array}\right) \quad A^{\mathrm{T}} = \left(\begin{array}       {cccc}
        1 & 6 & 11 &  16 \\
        2 & 7 & 12 & 17 \\
        3 & 8 & 13 & 18 \\
        4 & 9 & 14 & 19 \\ 
        5 & 10 & 15 & 20 
    \end{array}\right)$
\end{itemize}
\end{example}
\begin{definition}{Matrice Trasposta Coniugata}{}
    Sia $A$ una matrice avente come entrate valori complessi. Deifiniamo la sua \textbf{matrice trasposta coniugata}, rappresentata dal simbolo $A^{\dagger}$, come la matrice ottenuta effettuando la trasposta e scambiando ogni valore con il suo comlesso coniugato.
\end{definition}
\begin{example}{}{}
    $A = \left(\begin{array}{cc}
        3+9i & 2+i \\
        7-6i & 1-3i
    \end{array}\right) \quad A^{\dagger} = \left(\begin{array}       {cc}
        3-9i & 7+6i \\
        2-i & 1+3i \\
    \end{array}\right)$
\end{example}
\begin{definition}{Matrici Unitarie}{}\label{matUnit}
    Sia $U$ una matrice quadrata complessa. Definiamo $U$ come una \textbf{matrice unitaria} se:
    \begin{equation*}
        U^{\dagger}U = \mathbb{1} = UU^{\dagger}
    \end{equation*}
    dove $U^{\dagger}$ è la matrice trasposta coniugata di $U$ e $\mathbb{1}$ è la matrice identità.
\end{definition}
\begin{fact}{}{}
    Sia $U$ una matrice unitaria. Allora abbiamo che:
    \begin{equation*}
        \Vert Uv \Vert = \Vert v \Vert
    \end{equation*}
\end{fact}
\subsection{Notazione Dirac}