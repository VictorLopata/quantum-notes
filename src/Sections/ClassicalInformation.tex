\section{Informazione Classica}
Per comprendere al meglio come funziona l'informazione e la computazone quantistica, è bene avere le idee chiare su come funziona quella classica.
\subsection{Sistemi Singoli}
Sia $X$ un sistema fisico che memorizza l'informazione. $X$ può stare in un numero \textbf{finito di stati}. Definiamo anche $\Sigma$ come l'insieme finito degli stati che $X$ può assumere.

\begin{example}{}{}
    Ad esempio possiamo pensare ad $X$ come un bit, quindi $\Sigma = \{0,1\}$.
\end{example}

\begin{definition}{Stato Probabilistico}{}
    Sia $X$ un sistema e $\Sigma$ il suo insieme di stati. Definiamo gli \textbf{stati probabilistici} di $X$ se associamo ad ogni stato una \textbf{probabilità} tale che:
    \begin{itemize}
        \item $0 \leq p(\sigma) \leq 1$ per ogni $\sigma \in \Sigma$
        \item $\sum_{\sigma \in \Sigma} p(\sigma) = 1$
    \end{itemize}
\end{definition}

Possiamo rappresentare gli \textbf{stati probabilistici} come vettori, chiamati anche \textbf{vettori probabilistici}.
\begin{example}{}{}
    Sia $X$ il sistema che rappresenta un bit. Con probabilità $\frac{3}{4}$ $X$ assume lo stato di $0$, con $\frac{1}{4}$ assume $1$. Allora possiamo rappresentare questo stato attraverso il seguente vettore:
    \begin{equation*}
         \left(\begin{array}{c}
             \frac{3}{4}  \\ \\
             \frac{1}{4} 
        \end{array}\right)
    \end{equation*}
    dove la prima entrata corrisponde la probabilità che $X$ assuma lo stato $0$, la seconda entrata corrisponde alla probabilità che $X$ assuma lo stato $1$.

\end{example}

È comodo utilizzare la Dirac Notation (Sezione 1.5) per esprimere uno stato probabilistico.

\begin{definition}{Standard Basis Vectors}{}
    Definiamo come \textbf{Standard Basis Vectors} i vettori che hanno tutte le entrate $0$ eccetto una singola entrata avente $1$. Sono utili per rappresentare gli stati classici.
\end{definition}
In particolare, per il nostro sistema binario, gli standard basis vectors sono $|0\rangle$, corrispondente a 
$\left(\begin{array}{c}
         1  \\
         0
    \end{array}\right)$
e $|1\rangle$, corrispondente a $\left(\begin{array}{c}
         0  \\
         1
    \end{array}\right)$.

\begin{fact}{}{}
    Ogni vettore probabilistico può essere espresso unicamente come una \textbf{combinazione lineare} degli standard basis vectors.
\end{fact}
\begin{example}{}{}
    \begin{equation*}
        \left(\begin{array}{c}
         \frac{3}{4}  \\ \\
         \frac{1}{4}
    \end{array}\right) = \frac{3}{4} |0\rangle +  \frac{1}{4} |1\rangle
    \end{equation*}
\end{example}
\subsubsection{Misurazione di stati probabilistici}
\subsubsection{Operazioni deterministiche}
\subsubsection{Operazioni probablistiche}
\subsubsection{Composizione di operazioni probabilistiche}
\subsection{Sistemi Multipli}
\subsubsection{Stati Classici}
\subsubsection{Stati Probabilistici}
\subsubsection{Misurazione di stati probabilistici}
\subsubsection{Operazioni sugli stati probabilistici}
